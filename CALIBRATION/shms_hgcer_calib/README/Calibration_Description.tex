\documentclass[12pt,letterpaper]{article}

\usepackage[T1]{fontenc}
\usepackage{lmodern}
\usepackage{setspace}
\usepackage{amsmath}
\usepackage{amssymb}
\usepackage{graphicx}
\usepackage{fullpage}

\doublespacing

\begin{document}

\title{SHMS Heavy Gas Cherenkov (HGC) Detector Calibration Procedure}
\author{Ryan Ambrose -- gra353@uregina.ca}
\date{}
\maketitle

\section{How to Perform a Calibration}
From the main directory (hallc\_replay/CALIBRATION/shms\_hgc\_calib) run the script Calibration.C (in terminal type: root Calibration.C). The terminal will then ask for a run number; once entered the script will output three canvases for each photomultiplier tube (PMT): heavy gas Cherenkov (HGC) spectra after several cuts to determine single photoelectron (SPE) peak, calibration of full ADC spectra with calibration effectiveness, and a second round of calibration based off the slope of the previous. Calibration was successful if slope of the line is approximately 1.

\section{Data Cuts and Histogram Creation}

After the raw cdaq data are replayed using replay\_shms.C, it is fed into a script designed to sort the data event-by-event. For each event, the script extracts the pulse amplitude, integral, and timing information from the HGC, shower/preshower, noble gas Cherenkov (NGC), Aerogel, and various other quantities from the trigger, drift chamber, and hodoscopes for timing information. Each event corresponds to a single trigger, and the first cut performed is the requirement that the quantity $ntracks$ is equal to one, i.e. there is one and only one good track for the event. This is to eliminate the large number of signals coming from sources outside of the beam path. Next, several histograms are filled for troubleshooting and verifying the correct operation of the script. To reduce run time, the filling of histograms for other detectors may be omitted, since they have little relevance for the HGC. A quantity which will be useful is the pulse integral for each detector for each event, for particle identification. \\

For the actual HGC loop, it is important to understand how the data is output from the replay script. The leaves of interest are ``P.hgcer.goodAdcPulseInt'', ``P.hgcer.goodAdcPulseAmp'', and ``P.hgcer.goodAdcPulseTime'', which have inherent in them cuts already. They require that the FPGA did not fail and the hit must have occured within the prompt FADC timing peak (500. < adcTime < 2500.). The quantity PulseInt refers to the pulse integral, summing the voltage at each sampling point of the pulse. PulseAmp refers to the pulse amplitude (maximum voltage achieved) and PulseTime refers to the time it takes the pulse to reach the half-maximum point from the initial sampling. Each of these leaves are then indexed into arrays of length four corresponding to each PMT. The resulting histograms are highly sparsified due to a signal of 0.0 being filled for each event the HGC doesn't record any measurement (such as tracks not going through the HGC). \\

Two major cuts are then performed: on particle ID and particle location. Particle ID is determined by comparing the pulse integral between the NGC and the sum of the shower and pre-shower (SHWR/PSHWR). Naturally, electrons deposit more energy when they traverse a medium which dictates the division between electrons and pions. Particle location is determined by using the tracking replayed data and the geometric location of the HGC mirrors. A small angle of deflection from the optical axis is assumed, simplifying the calculation of particle location in the HGC to:
\begin{eqnarray*}
y_{HGC} = y_{Focal Plane} + y'_{Focal Plane}* z_{HGC} \\
x_{HGC} = x_{Focal Plane} + x'_{Focal Plane}* z_{HGC} \\
z_{HGC} = 156.27\text{cm (From focal plane)}
\end{eqnarray*} 
The mirror plane of the HGC is divided into four quadrants based of detector geometry. Individual histograms are filled for each type of particle in each quadrant, indexed by PMT, and these are used in the actual calibration script. 

\section{Calibration}

To perform the calibration, it is assumed the ADC pulse integral and corresponding number of photoelectrons have a linear relationship. In the first step ADC histograms for the HGC with cuts on particle ID and location are used to visolate SPE events. As a note, each PMT looking at its own quadrant (signal PMT 1 gets from quadrant 1) is ignored since there are many fewer SPE events. The ROOT class TSpectrum, specifically the method search(), is used to locate the peak of the pedistle and SPE. The data is fit with a sum of two Gaussian distributions with their mean restricted to the prediction from TSpectrum within $\pm$10 ADC channels. The resultant mean for the SPE for electrons is then averaged and used as a first guess for a calibration. Pion events are not used due to typically low statistics. \\

Once the calibration is obtained the full spectra, for each PMT, is scaled from ADC channel to photoelectron and normalized by its integral. This calibrated spectra is then fitted by a Poisson distribution on the right hand side to remove the non-Gaussian characteristics of that larger photoelectron peaks (remove signal from approximately 5.5 photoelectrons and onward). Once the Poisson characteristics are subtracted, the final spectrum is fit with a sum of three Gaussians to ensure the peaks correspond to integer photoelectron values. As a verification, a plot of photoelectron position is created to check for linearity. After this first iteration, the calibration constant is multiplied by the linear slope and the process is repeated. Both calibration constants are outputted so that the better fit can be determined.
\end{document}
