\documentclass[12pt,a4paper]{article}

\usepackage[T1]{fontenc}
\usepackage{lmodern}
\usepackage{setspace}
\usepackage{amsmath}
\usepackage{amssymb}
\usepackage{graphicx}
\usepackage{fullpage}

\doublespacing

\begin{document}

\title{SHMS Heavy Gas Cherenkov (HGC) Detector Calibration Procedure}
\author{Ryan Ambrose}

\maketitle

\section{Data Cuts and Histogram Creation}

After raw cdaq data is replayed using replay\_shms.C it is fed into a script designed to sort the data event by event. For each event the script extracts the pulse amplitude, integral, and timing information from the HGC, shower/preshower, NGC, Aerogel, and various other quantities from the trigger, drift chamber, and hodoscopes for timing information. Each event corresponds to a single trigger and the first cut performed is the requirement that the quantity $ntracks$ is equal to one, i.e. there is one and only one good track for the event. This is to eliminate the large number of signals coming from sources outside of the beam path. Next several histograms are filled for troubleshooting and verifying the correct operation of the script. To reduce run time the filling of histograms for other detectors may be omitted, since they have to relevance for the HGC. A quantity which will be useful is the pulse integral for each detector for each event, for particle identification. \\

For the actual HGC loop it is important to understand how the data is output from the replay script. The leaves of interest are ``P.hgcer.goodAdcPulseInt'', ``P.hgcer.goodAdcPulseAmp'', and ``P.hgcer.goodAdcPulseTime'' which have inherent in them cuts already. They require that the FPGA did not fail and the hit must have occured within the prompt FADC timing peak (500. < adcTime < 2500.). Two major cuts are then performed, on particle ID and particle location. 

\end{document}
